%	This is written by Zhiyang Ong to indicate guidelines that members of open source software and/or hardware projects shall follow.

%	The MIT License (MIT)

%	Copyright (c) <2014> <Zhiyang Ong>

%	Permission is hereby granted, free of charge, to any person obtaining a copy of this software and associated documentation files (the "Software"), to deal in the Software without restriction, including without limitation the rights to use, copy, modify, merge, publish, distribute, sublicense, and/or sell copies of the Software, and to permit persons to whom the Software is furnished to do so, subject to the following conditions:

%	The above copyright notice and this permission notice shall be included in all copies or substantial portions of the Software.

%	THE SOFTWARE IS PROVIDED "AS IS", WITHOUT WARRANTY OF ANY KIND, EXPRESS OR IMPLIED, INCLUDING BUT NOT LIMITED TO THE WARRANTIES OF MERCHANTABILITY, FITNESS FOR A PARTICULAR PURPOSE AND NONINFRINGEMENT. IN NO EVENT SHALL THE AUTHORS OR COPYRIGHT HOLDERS BE LIABLE FOR ANY CLAIM, DAMAGES OR OTHER LIABILITY, WHETHER IN AN ACTION OF CONTRACT, TORT OR OTHERWISE, ARISING FROM, OUT OF OR IN CONNECTION WITH THE SOFTWARE OR THE USE OR OTHER DEALINGS IN THE SOFTWARE.

%	Email address: echo "cukj -wb- 23wU4X5M589 TROJANS cqkH wiuz2y 0f Mw Stanford" | awk '{ sub("23wU4X5M589","F.d_c_b. ") sub("Stanford","d0mA1n"); print $5, $2, $8; for (i=1; i<=1; i++) print "6\b"; print $9, $7, $6 }' | sed y/kqcbuHwM62z/gnotrzadqmC/ | tr 'q' ' ' | tr -d [:cntrl:] | tr -d 'ir' | tr y "\n"

%%%%%%%%%%%%%%%%%%%%%%%%%%%%%%%%%%%%%%%%%%%%%%



%%%%%%%%%%%%%%%%%%%%%%%%%%%%%%%%%%%%%%%%%%%%%%
%	Preamble.
\documentclass[letter,12pt]{article}
%%%%%%%%%%%%%%%%%%%%%%%%%%%%%%%%%%%%%%%%%%%%%
%
%	Importing LaTeX source files, without quoting the ".tex" extension.
%
%%%%%%%%%%%%%%%%%%%%%%%%%%%%%%%%%%%%%%%%%%%%%

%%%%%%%%%%%%%%%%%%%%%%%%%%%%%%%%%%%%%%%%%%%%%
%	File containing the LaTeX preamble.
% This is written by Zhiyang Ong as the preamble for all his LaTeX documents.
%
% It includes a list of LaTeX packages that he commonly uses to typeset LaTeX documents.

%	The MIT License (MIT)

%	Copyright (c) <2014> <Zhiyang Ong>

%	Permission is hereby granted, free of charge, to any person obtaining a copy of this software and associated documentation files (the "Software"), to deal in the Software without restriction, including without limitation the rights to use, copy, modify, merge, publish, distribute, sublicense, and/or sell copies of the Software, and to permit persons to whom the Software is furnished to do so, subject to the following conditions:

%	The above copyright notice and this permission notice shall be included in all copies or substantial portions of the Software.

%	THE SOFTWARE IS PROVIDED "AS IS", WITHOUT WARRANTY OF ANY KIND, EXPRESS OR IMPLIED, INCLUDING BUT NOT LIMITED TO THE WARRANTIES OF MERCHANTABILITY, FITNESS FOR A PARTICULAR PURPOSE AND NONINFRINGEMENT. IN NO EVENT SHALL THE AUTHORS OR COPYRIGHT HOLDERS BE LIABLE FOR ANY CLAIM, DAMAGES OR OTHER LIABILITY, WHETHER IN AN ACTION OF CONTRACT, TORT OR OTHERWISE, ARISING FROM, OUT OF OR IN CONNECTION WITH THE SOFTWARE OR THE USE OR OTHER DEALINGS IN THE SOFTWARE.

%	Email address: echo "cukj -wb- 23wU4X5M589 TROJANS cqkH wiuz2y 0f Mw Stanford" | awk '{ sub("23wU4X5M589","F.d_c_b. ") sub("Stanford","d0mA1n"); print $5, $2, $8; for (i=1; i<=1; i++) print "6\b"; print $9, $7, $6 }' | sed y/kqcbuHwM62z/gnotrzadqmC/ | tr 'q' ' ' | tr -d [:cntrl:] | tr -d 'ir' | tr y "\n"

%%%%%%%%%%%%%%%%%%%%%%%%%%%%%%%%%%%%%%%%%%%%%%%%%%

% Importing some standard LaTeX packages.

% To enable standard LaTeX processing for graphics. It enables PDF, JPEG, PNG, and TIFF graphics files to be included in the LaTeX document.
\usepackage{graphicx}
% For better typesetting of mathematical expressions, from the American Mathematical Society (AMS).
\usepackage{amsmath}
% For better typesetting of mathematical expressions, from the American Mathematical Society (AMS). This package includes mathematical symbols for the ``amsmath'' package.
\usepackage{amssymb}
% For better typesetting of mathematical proofs (for theorems and colloraries), from the American Mathematical Society (AMS).
\usepackage{amsthm}
%	Create definitions for new theorems, axioms, colloraries.
	\newtheorem{theorem}{Theorem}[section]
	\newtheorem{axiom}{Axiom}[section]
	\newtheorem{corollary}{Corollary}[section]
	\newtheorem{lemma}{Lemma}[section]
	\newtheorem{Rule}{Rule}[section]
	\newtheorem{law}{Law}[section]
	\newtheorem{principle}{Principle}[section]
% To change the style of newly defined theorems.
%		\usepackage{theorem}


% For better typesetting of tables (and arrays).
\usepackage{array}
% For creating tables without vertical separators.
%		\usepackage{booktabs}
% To control line spacing in LaTeX documents.
\usepackage{setspace}
% To modify the spacing between words and letters.
%		\usepackage{microtype}
% To change the dimensions of the page(s).
%\usepackage[margin=1.5cm,vmargin={0pt,1cm},nohead]{geometry}
\usepackage[margin=1.5cm,vmargin={1.5cm,2cm}]{geometry}
% Use the packages needed to typeset algorithms. I can also use the combined ``algorithms'' bundle.
\usepackage{algorithm}
\usepackage{algorithmic}
% The listings package is a source code printer for LaTeX. You can typeset stand alone files as well as listings with an environment similar to verbatim as well as you can print code snippets using a command similar to \verb. Many parameters control the output and if your preferred programming language isn�t already supported, you can make your own definition.
\usepackage{listings}
% Use the ``clrscode3e'' LaTeX package to typeset algorithms like CLRS
%	\usepackage{/data/others/notes/clrscode3e}
%\usepackage{/Users/zhiyang/Documents/rapporti/grappanotes/clrscode3e}
\usepackage{/Users/zhiyang/Documents/rapporti/paneer-notes/others/clrscode3e}
%\usepackage{/data/others/grappanotes/clrscode3e}
% Use the ``algpseudocode'' LaTeX package to typeset algorithms -- Alternate solution, not preferred
%\usepackage{algpseudocode}
% Alternative packages for typesetting algorithms.
%\usepackage{algorithm2e}
%\usepackage{algorithmicx}
%\usepackage{program}
%	To check for syntax errors in my LaTeX document.
\RequirePackage[l2tabu, orthodox]{nag}

% Concatenate adjacent references together when typeset.
% That is, cite{ref1,ref2,ref3,ref4} can appear as [12-15], instead of [12] [13] [14] [15]
\usepackage{cite}
% For automatic insertion of cross-referencing words, such as fig. for figures and eq. for equations.
%		\usepackage{cleveref}

% LaTeX support for Metafont and MetaPost logos.
\usepackage{mflogo}













% How to typeset single and double quotes for feet and inches?
% For feet, use [FEET]\textasciiacute
% For inches, use [INCHES]\textacutedbl
% For feet and inches, use [FEET]\textasciiacute\ [INCHES]\textacutedbl; force a character space between the single quote for feet and the height of the object in inches
% Don't use \textceltpal as a single quote to represent height in feet, or double \textceltpal (two concatenated \textceltpal) as a double quote to represent height in inches
% For double quotes, don't use two single quotes provided by the default settings of LaTeX. The resultant double quotes will be curly.

% The tipa package is for Phonetic Symbols -- I wanna use the \textceltpal symbol to represent a single quote, instead of using the generic ``curly'' single quote from \LaTeX (Table 10, pp.10)
\usepackage{tipa}
% The textcomp package is for Diacritics -- I wanna use the \textacutedbl symbol to represent a double quote (Table 28, pp.17), instead of using the generic ``curly'' double quotes from \LaTeX; however, when this symbol is used, I must force a character space to exist after the symbol by using the backslash followed by a character space. This package also provides the symbol for Copyleft, \textcopyleft, which is not available in LaTeX by default, and provides better looking symbols for: copyright, registered, and trademark (Table 33, pp.18). Also, it provides symbols for: \textcelsius, \textmho, \textmu, \textohm (Table 201, pp.67). It also provides symbols for Genealogical Symbols (Table 253, pp78), such as \textborn, \textdivorced, \textmarried, \textdied, and \textleaf (symbol of a leaf)... Its symbol for the Euro, EU currency, is \texteuro
\usepackage{textcomp}
% Look at \url{http://www.ctan.org/tex-archive/info/symbols/comprehensive/symbols-a4.pdf} for a list of symbols that can be used in LaTeX and its packages. Table 280, pp.88, deals with Symbol Name Clashes; hence, if the same command name refers to multiple symbols, the symbol-conflict resolution abides by this.
% In particular, check out the gensymb package (Table 197, page 67) for symbols defined to work in both math and text modes, such as \celsius, \micro, \degree, and \ohm.
% Also, check out the wasysym package (Table 198, page 67) for electrical symbols, such as that of alternating current (AC); it also provides symbols for \female, and \male (Table 212, pp.70); it also has symbols for ``Xs and Check Marks,'' which are checked boxes, \CheckedBox, squares, \Square, and crossed boxes (boxes filled with a cross), \XBox (Table 232, pp.73); it also has symbols for a clock, \clock, a Simley, \smiley, diameter, \diameter, lightning, \lightning, sun, \sun, and a tick or check mark \checked (symbol to indicate that something is correct), and a bell, \bell (Table 254, pp.78); it also has symbols for left and right turns (Table 256, pp.78), \leftturn and \rightturn; this package (Table 256, pp.78) and the arev package (Table 257, pp.78) can be used to typeset music symbols, along with Table 182, pp.62; it also has symbols for Navigation (Table 261, pp.79), such as \Forward, \RewindToStart, and \ForwardToIndex; it also has symbols for laundry (Table 262, pp.80); it also has the symbol for a heart, \Heart (Table 263, pp.80).
% In addition, check out the ifsym package (Table 199, page 67) for pulse diagram symbols; it also has symbols for weather (Table 266, pp.80), alpine and mountain climbing, such as \Summit, \Mountain, \IceMountain, \VarMountain, \Flag, \FilledHut, \Hut, \Village, and \Tent (Table 267, pp.81); it also has different symbols for clocks, such as \Interval, \StopWatchEnd, \VarClock, \showclock (to indicate the time) (Table 268, pp.81); it also has symbols for fire, letter, telephone, dice, \PaperPortrait, and \PaperLandscape. Also, has symbol for the cross to indicate that something is incorrect
\usepackage{ifsym}
% Besides, check out the keystroke package (Table 208, page 69) for symbols of Computer Keys, such as Alt, Ctrl, Del, Page down, Esc, Enter, Shift, Space Bar, and Up Arrow.
% From the dingbat package (Table 225, page 72), it has symbols for Fists, such as \rightthumbsdown and \rightthumbsup.
%\usepackage{dingbat}
% From the pifont package (Table 234, page 73), it has symbols for Circled Numbers, such as any digit that is circled, where the space in the circle can be shaded black.
% From the dictsym package (Table 277, page 84), it has symbols for dictionaries, and indicates which type of dictionary will define this term - say a medical, technical, mathematical, or judical dictionary
% The simpsons package can be used to indicate characters from {\it The Simpsons} (Table 278, pp.85)
% The symbol for quadruple integrals \iiiint is available as an AMS Variable-sized Math Operator, or I can use this symbol from the packages txfonts, pxfonts, esint, or MnSymbol 










% The marvosym package (Table 210, page 69) is for Communication Symbols, such as \Email, \fax, \FAX (Preferred), \Letter, \Mobilefone, and \Telefon; it also has the symbol for the Cross to represent Christianity, \Cross (Table 263, pp.80); it also has symbols for checked boxes, \Checkedbox, crossed boxes (boxes marked with a cross), \Crossedbox, bicycles, \Bicycle, clocks, \Clocklogo, the industry, \Industry, taking notes manually with pen/pencil and paper, \Writinghand, coffee, \Coffeecup, providing information or important note, \Info (Table 249, pp.76)... In addition, it has the symbols for the Euro (EU currency), \EUR (OK), \EURdig (OK), \EURtm, \EURcr
\usepackage{marvosym}
% From the bbding package (Table 226, page 72), it has symbols for Fists, such as \HandPencilLeft; it also has symbols for the Cross to represent Christianity, such as \Cross and \CrossOpenShadow (Table 228, pp.72); Use of the symbol \Cross has bugs; bugs exist in the package, as it fails to correctly overwrite the \Cross symbol; also has the peace symbol, \Peace. 
%\usepackage{bbding}
% The skak contains a cross, incorrect symbol that I can use to indicate that something is wrong, e.g. \markera or \weakpt
\usepackage{skak}
% Package to enable the use of a strikeout/strikethrough font with LaTeX. To use the strikeout/strikethrough font, use the ``sout'' LaTeX command, or tag,  to ``strike through'' text. E.g., \sout{Bill Clinton} G.W. Bush is the pres.
\usepackage{ulem}
% The eurosym package has the symbols for the Euro (EU currency), \geneuro, \geneuronarrow, \geneurowide, \officialeuro (GOOD)
\usepackage{eurosym}











% Create fancy headers and footers for this document
\usepackage{fancyhdr}
\setlength{\headheight}{15.2pt}
\pagestyle{fancy}
% Headers for the document
\lhead{}
%\lhead{Zhiyang Ong}
%\rhead{\today}
% Footers for the document
\lfoot{Zhiyang Ong}
\cfoot{}
\rfoot{\thepage}

% The following does not work, since it does not differentiate between odd and even pages. Hence, the last odd/even command will overwrite the previous even/odd command
%\fancyhf{}
%\fancyhead[LE]{Author's DFM}
%\fancyhead[LO]{\today EDA}
%\fancyfoot[LE]{\thepage USC}
%\fancyfoot[RO]{\thepage Adel}


% Allow for multi-line comments
\usepackage{verbatim}




% Commands for using the package for hyperlinks. Includes the package ``url''.
\usepackage[pdftex,
	pdftitle={Graphics and Color with LaTeX},
	pdfauthor={Patrick W Daly},
	pdfsubject={Importing images and use of color in LaTeX},
	pdfkeywords={LaTeX, graphics, color},
	pdfpagemode=UseOutlines,bookmarks, bookmarksopen,
	pdfstartview=FitH, colorlinks, linkcolor=blue, citecolor=blue, urlcolor=red,
]{hyperref}
\hypersetup{colorlinks, linkcolor=blue}







% Create a glossary for symbols and terms in this document
% The following attempt failed
%\makeglossaries

% The following attempt failed
%%%%%%%%%%%%%%%%%%%%%%%%%%%%%\makeglossary
%\usepackage{supertabular}
%\newcommand{\glossaryname}{Symbols Index}
%\newenvironment{theglossary}
%    {\section*{Symbols Index}
%      \begin{supertabular}{ll}}
%    {\end{supertabular}
%}
%\newcommand{\printglossary}{\InputIfFileExists{zhiyang_ong.glo}{}{\section*{Symbols Index - File not found}}}

% Another failed attempt at creating a glossary
%\input{gatech-thesis-gloss.sty}
%\usepackage{gatech-thesis-gloss}
%\glossfiles{zhiyang_ong.glo}

% Create the glossary
\usepackage{nomencl}
\makenomenclature


% Enable captions to be modified.
%\usepackage{caption}
% Addition support for colored text.
%\usepackage{color}
% Enable the insertion of PDF/PS files/documents.
		\usepackage{pdfpages}
% To rotate objects, including tables.
		\usepackage{rotating}
% To define multiple floats (figures and tables), with individual captions and labels, within one environment.
		\usepackage{subfig}
% For a modular LaTeX document with multiple files (including the ``root file''), it allows the a non-empty subset of the ``child files'' to be typeset without having to typeset the ``root file'' (and/or the other ``child files'').
		\usepackage{subfiles}
% To annotate the LaTeX document with to-do notes.
		\usepackage[colorinlistoftodos]{todonotes}
% To insert images surrounded by text.
		\usepackage{wrapfig}
% To create trees, graphs, (commutative) diagrams, and similar things. Reference: Wikibooks contributors, ``\LaTeX/Xy-pic,'' in {\it \LaTeX}, Wikibooks: Open books for an open world, Wikimedia Foundation, San Francisco, CA, June 5, 2005. Available online at: \url{http://en.wikibooks.org/wiki/LaTeX/Xy-pic}; last accessed on December 25, 2013.		=> This package seems to have bugs in it. If I use this package, my document will not typeset properly. I have tried to use it successfully in other documents. It does not seem to be compatible with 
%\usepackage{xypic}
% Package for SI units.
\usepackage{siunitx}












%	For typesetting the symbol: \AE.
\usepackage[T1]{fontenc}
\usepackage[utf8]{inputenc}
\usepackage{lmodern}












%%%%%%%%%%%%%%%%%%%%%%%%%%%%%%%%%%%%%%%%%%%%%%%%%%
% Other helpful hints:

% To use the italic and bold font concurrently, try this: {\itshape Review the {\bfseries updated} training log}

% To use the symbol for summation, which is the capital-sigma notation, with proper super- and sub- fixes, try: $\displaystyle\sum_{i = -1}^{m} \frac{log_2 n_i}{T_i}$

% Make sure that I include the following so that I can cite references properly: \usepackage{cite}. This allows references to be included as [1-10], rather than [1], [2], [3], [4], [5], [6], [7], [8], [9], [10]

% Colors that appear well in PDF format for LaTeX text include: red, blue, and magenta

% Use \scriptsize, instead of \textsc, \sc, or \schape to use small caps. Currently, I cannot use \textsc, \sc, or \schape to write in small caps on my MacBook Pro laptop.

% The Typewriter font cannot be used concurrently with the bold font. That is, the following cannot be used: {\tt \bf text}, AND \texttt{\textbf{text}}

% Use \LaTeX for LaTeX; B{\scriptsize IB}\TeX to indicate the symbol for BibTeX; \texttrademark for trademarks; \MF for Metafont; and \MP for MetaPost




%%%%%%%%%%%%%%%%%%%%%%%%%%%%%%%%%%%%%%%%%%
%																%
%	Default colors that I can use with \LaTeX:								%
%	1) red														%
%	2) green														%
%	3) blue														%
%	4) yellow														%
%	5) cyan														%
%	6) magenta													%
%	7) black														%
%	8) white														%
%																%
%%%%%%%%%%%%%%%%%%%%%%%%%%%%%%%%%%%%%%%%%%


% Partial list of ``the 68 predefined internal colors of the {\tt dvips} PostScript driver'' \cite{Kopka04} that I can use for changing the color of text ... Use bold font for the text
%YellowOrange
%RoyalBlue
%DarkOrchid
%ForestGreen
%OliveGreen
%Mulberry
%ProcessBlue
%RubineRed
%VioletRed
%WildStrawberry
% E.g., try: \textcolor{VioletRed}{\bf hello world}

% As for changing the background color of text, choose a light colored background to make the text stand out in black colored bold font; see \url{oregonstate.edu/~peterseb/tex/samples/docs/color-package-demo.pdf} for a list of colors
% E.g., try: \colorbox{Apricot}{\bf hello world}





% definition of new \LaTeX command for the citation: \cite{Cimatti08} and \cite{Barrett09}
% This allows mathematical/logic symbols to be typeset with the font ``Zapf Chancery'' in ``\LaTeX\ math mode''. To typeset symbols in such font, try: \mathpzc{ABCdef123}
\DeclareMathAlphabet{\mathpzc}{OT1}{pzc}{m}{it}

%%%%%%%%%%%%%%%%%%%%%%%%%%%%%%%%%%%%%%%%%%%%%
% Start of document
\begin{document}
\title{Guidelines for Collaboration}
\date{\today}
\author{Zhiyang Ong \thanks{Email correspondence to: \href{mailto:ongz@acm.org}{\Email\ ongz@acm.org}}}
\maketitle


\begin{abstract} 
This is a set of guidelines for development/design processes and practices, and conduct while collaborating on open source projects. It also includes guidelines for creating a shared {\sc Bib}\TeX\ database.
\end{abstract}


%%%%%%%%%%%%%%%%%%%%%%%%%%%%%%%%%%%%%%%%%
%	Create the table of contents
\tableofcontents
%	Start the numbering of chapters from 1, instead of 0.
%\setcounter{chapter}{1}
%	Increase the depth of each section in the Table of Contents to 4.
\setcounter{secnumdepth}{4}



%%%%%%%%%%%%%%%%%%%%%%%%%%%%%%%%%%%%%%%%%%%
\section*{Revision History}
\label{sec:RevisionHistory}
\addcontentsline{toc}{section}{Revision History}


Revision history: \vspace{-0.3cm}
\begin{enumerate} \itemsep -4pt
\item Version 1, October 2, 2014. Initial version of the guideline (for another project).
\item Version 1.1, December 23, 2014. Version ported for this boilerplate code project.
\item Version 2, October 20, 2015. Added guidelines for {\tt Doxygen}-supported, {\tt Javadoc}-based coding standard. This coding standard is also known as coding style, coding style guide, coding guideline, coding scheme, code convention, code documentation guideline, programming guideline, or programming style.
\item Version 2.1, October 21, 2015. Finished guidelines for {\tt Doxygen}-supported, {\tt Javadoc}-based coding standard for {\it C++}.
\item Version 2.2, June 4, 2016. Finished section for additional guidelines: to include documentation using {\it Markdown}, and tools for software development, integrated circuit and cyber-physical system design, and documentation.
\item Version 3, November 3, 2016. Added guidelines for: documenting {\tt GNU Octave} and {\sc Matlab} code, in order to facilitate documentation generation using {\it Texinfo} \cite{WikipediaContributors2016h,Stallman2016,Stallman2016a,Stallman2016b}; sharing of source code, design files, sets of benchmarks, data sets, and documentation on online repositories \cite{figshareLLPstaff2016,GitHubStaff2016}; and added section on exception safety.
\item Version 3.1, November 4, 2016. Fixed references for indent style conventions.
\item Version 3.2, December 20, 2016. Update guidelines for conduct.
\item Version 3.3, February 3, 2017. Update information about usage of {\it GitHub}'s services.
\item Version 3.4, March 11, 2017. Update information on naming convention.
\item Version 3.5, October 9, 2017. Update guidelines on commenting/writing code.
\item Version 3.6, December 24, 2017. Fix grammatical error in a sentence.
\item Version 3.7, January 25, 2018. Added suggestions for software architecture of my computer programs.
\item Version 3.8, January 31, 2018. Added information about coding style guideline for different computer languages, and also about online repositories that facilitate research reproducibility, replicability, and repeatability.
\item Version 3.9, February 1, 2018. Added information about developing software in a {\it Python}ic style.
\item Version 4.0, June 8, 2018. Added information about specifying (co-)authors' full name, and research reproducibility, and other best practices from software development, and embedded/cyber-physical system and integrated circuit design.
\item Version 4.1, September 19, 2018. Updated ACM Code of Ethics and Professional Conduct; added The Joint {ACM/IEEE-CS} Software Engineering Code of Ethics and Professional Practice; and updated guidelines on exception handling.
\item Version 4.2, September 21, 2018. Added acknowledgements, shout outs, for people who helped me with automated regression testing. And, refactored document.
\item Version 4.3, September 22-25, 2018. Added references on agile SoC design, and hardware/VLSI/RTL/HDL refactoring.
\item Version 4.4, December 29-30, 2018. Added template information for the {\tt Annote} and {\tt Howpublished} {\sc Bib}\TeX\ fields to help me identify particular information about the document/publication.
\item Version 4.5, February 11, 2019. Extend recommendations and suggestions to projects in data science and (applied) machine learning.
\item Version 4.6, February 14, 2019. Add references for {\it Git} and its substitute, {\it Mercurial SCM}, for distributed version/revision control (or software configuration management). Also, add ``Acknowledgments'' and ``Revision History'' sections to the ``Table of Contents''.
\item Version 4.7, March 13, 2019. Add comment on how to include mathematical expressions in {\it Markdown} documents. In addition, fix minor errors.
\item Version 4.8, January 22, 2020. Fix B{\scriptsize IB}\TeX\ key error (renamed for a publication), and added references regarding design by contract (or programming by contract, or contract programming) and Hoare logic.
\item Version 4.9, March 11, 2021. Fix minor errors in guidelines for creating shared B{\scriptsize IB}\TeX\ databases.
\item Version 5.0, June 20, 2021. Refactor \S\ref{ssec:TemplateInformationForTheAnnoteBibTeXField} to include information about publications in the {\tt Annote} field of {\sc Bib}\TeX\ entries.
\end{enumerate}









%%%%%%%%%%%%%%%%%%%%%%%%%%%%%%%%%%%%%%%%%%%
\section{Guidelines for Conduct}
\label{sec:GuidelinesforConduct}

Collaborators of open source software and/or hardware projects that we are involved in should follow the {\it Code of Conduct} of the {\it Institute of Electrical and Electronics Engineers} (IEEE) \cite{IEEE2014b,IEEE2014a,IEEE2014} and the {\it ACM Code of Ethics and Professional Practice} of the {\it Association for Computing Machinery} (ACM) \cite{Gotterbarn2018b,Gotterbarn2018a,Gotterbarn2018,Brinkman2016a,ACMCouncil1992,Brinkman2017,Brinkman2016,Wolf2016,Anderson1993}, including the ``The Joint {ACM/IEEE-CS} Software Engineering Code of Ethics and Professional Practice'' \cite{Gotterbarn1999,Gotterbarn1997}. Also, actions of discrimination are not acceptable \cite{IEEE2014c}; we should intentionally commit to inclusive diversity. An additional guideline is ``Dave Packard's 11 simple rules'' \cite{HewlettPackardCompany2012}. \\

%There are a significant amount of references for helping people to learn \LaTeX \cite{Voss2011,vanDongen2012,Syropoulos2003,Raymond2004,Mittelbach2004,Lamport1994,Krishnan2003,Krantz2001,Kottwitz2011,Koranne2011,Kopka2004,Knuth1999,Hoenig1998,Higham1998,Haralambous2007,Griffiths1997,Gratzer2007,Goossens2007,Goossens1999,Goossens1997,Diller1999,Bindner2011,Berry2009,UITCambridge2011,Scharrer2011,Pakin2008,Cormen2010,Syropoulos2004,Hamalainen2006} and related information/technologies.

In addition, when there is a dispute about which technology, algorithm, design paradigm/style/pattern, process, or methodology to use, follow the ``Code Wins Arguments'' philosophy \cite{Kushner2011,Zuckerberg2012}. Also, when considerable effort has been invested in an automated regression testing/verification infrastructure, do not be afraid to ``move fast and break things'' \cite{Fong2011,Evangelista2012}. \\

Lastly, we should adopt a mission-focused and value-based approach to participate in meetings and discussions for the project(s). We should be flexible/liberal enough to consider and explore viable alternate approaches to do things and solve problems \cite{Beedle2001,Beedle2001a}. Where disputes occur, a data-driven, fact-based approach based on the ``Code Wins Arguments'' philosophy should be used to resolve conflicts.

%%%%%%%%%%%%%%%%%%%%%%%%%%%%%%%%%%%%%%%%%%%
\section{Guidelines for Creating a Shared {\sc Bib}\TeX\ Database}
\label{sec:GuidelinesforCreatingaSharedBibTeXDatabase}

Guidelines for creating {\sc Bib}\TeX\ entries and the {\sc Bib}\TeX\ database, which is used for writing the paper, are given as follows: \vspace{-0.2cm}
\begin{enumerate} \itemsep -2pt
\item Each {\sc Bib}\TeX\ key should be unique: \vspace{-0.3cm}
	\begin{enumerate} \itemsep -2pt
	\item Check if your desired {\sc Bib}\TeX\ key already exists in the {\sc Bib}\TeX\ database: \vspace{-0.1cm}
		\begin{enumerate} %\itemsep -2pt
		\item If it does, do not add it to the  {\sc Bib}\TeX\ database.
		\item Else, add it to the  {\sc Bib}\TeX\ database.
		\end{enumerate}
	\item Use the following format for creating {\sc Bib}\TeX\ keys: [first] author's last name, appended by the year of publication. E.g., my first conference paper would have the {\sc Bib}\TeX\ key Ong2004. If the year of publication is not known, use an approximate year, with XY for the last 2 digits in the year (e.g., 20XY). Alternatively, if you cannot determine if it was published this millennium or the previous millennium (or much earlier), use UNKNOWN for the ``year''. For example, use KleinbergUNKNOWN (preferred for unknown millennium), or Smith20XY (for unknown year in the 21$^{st}$ century).
	\item Remove duplicate entries in the {\sc Bib}\TeX\ database. {\bf WARNING! Before doing this, perform a union operation on the fields of the {\sc Bib}\TeX\ entries. For example, if a {\sc Bib}\TeX\ entry has information that the other {\sc Bib}\TeX\ entry does not have, and vice versa, merge the information to one {\sc Bib}\TeX\ entry.}
	\item {\bf Rationale: Duplicate {\sc Bib}\TeX\ entries will cause problems in typesetting.}
	\item Regarding hash collision of {\sc Bib}\TeX\ keys, such as multiple instances of Gratz2014, distinguish them by appending a letter to them. E.g., use Gratz2014, Gratz2014a, Gratz2014b, Gratz2014c, and so on. If we run out of letters, append it with ``a'' followed by a number. The use of the letter ``a'' separates the year from the instance of {\sc Bib}\TeX\ key. That is, Gratz2014a2 tells me that it is the $29^{th}$ instance of Gratz2014, as opposed to Gratz201429.
	\item If possible, restrict the characters of each {\sc Bib}\TeX\ key to be alphanumeric. The year is always numeric, and is appended to the (first) author's last name. \vspace{-0.2cm}
		\begin{enumerate} \itemsep -2pt
		\item If the (first) author's last name has characters with diacritical marks, accents, or diacritics, trim the characters used to typeset the diacritical marks (or accents) from the (first) author's last name, and append the year of publication to it. E.g., {\it S{\v{o}}m{\'{e}}nz{\d{i}}} (year 2000) becomes {\it Somenzi2000}.
		\item If the (first) author's last name has characters that are not letters in English, anglicize those characters. We should avoid using the transliteration for a given non-English language, since such transliteration may not be standardized (for non-commonly spoken/used languages). Also, supporting letters from other languages is a tedious task. Hence, we can use the anglicized version of their last names instead.
		\end{enumerate}
	\end{enumerate}
\item If possible, use the full name for each author. \vspace{-0.3cm}
	\begin{enumerate} \itemsep -2pt
	\item We justify this as follows.
	\item When writing research publications, if we need to reduce the authors' first name to just their initial, we can use a script to transform their names.
	\item If we need to use their full names in the reference list and if we do not include their full names, we have to look up these references again in the future to include their full names.
	\end{enumerate}
\item For terms that should be typeset as is, place them in between braces (i.e., curly brackets). That is, put curly braces around acronyms and mixed-case names. \vspace{-0.3cm}
	\begin{enumerate} \itemsep -2pt
	\item For example, terms in upper or mixed cases (upper and lower cases), such as names (e.g., McMullen) and acronyms (e.g., SIGDA), place them in between braces (i.e., \{McMullen\} and \{SIGDA\}). This prevents the titles (or another {\sc Bib}\TeX\ field) from changing the term into lower case, with exception for the first term/word. E.g., ``ICCAD Update: A Report from SIGDA'' may typeset into ``ICCAD Update: A report from sigda''.
	\end{enumerate}
\item For special symbols that are typeset with \LaTeX\ in the {\tt math mode}, such as $\alpha$, place them in between a pair of dollar signs (i.e., \$$\backslash$alpha\$).
\item For each {\sc Bib}\TeX\ entry, check if all required fields are complete. See pages 8 and 9 in \S3.1 of \cite{Patashnik1988} for a list of {\sc Bib}\TeX\ entry types; alternatively, refer to the {\it Wikipedia} entry for {\sc Bib}\TeX, or \cite[\S12.2.1, pp. 230--231]{Kopka2004}. In this/these list(s), the required fields are listed for each {\sc Bib}\TeX\ entry.
\item For the {\tt Pages} field, ensure that all page ranges are indicated with double hyphens. E.g., ``Page = \{11-\--34\},''. This makes the page range look better.
\item For the {\tt Pages} field, ensure that multiple pages and/or page ranges are separated by commas. E.g., ``Page = \{11-34, 57, 88, 109-\--187\},''.
\item For books and journal articles that have an associated digital object identifier (DOI) \cite{InternationalDOIFoundationStaff2017}, ensure that the {\tt Doi} field is included in the {\sc Bib}\TeX\ entry with the DOI of the publication. This makes it easier for people to access the Web page for the book or journal/conference paper.
\item Stylistic validation of the references can be carried out as follows: \vspace{-0.3cm}
	\begin{enumerate} \itemsep -2pt
	\item Include all {\sc Bib}\TeX\ keys in one citation in your \LaTeX\ document.
	\item Typeset the \LaTeX\ document.
	\item Check that the font and style of the reference list is correct.
	\item If there are errors, correct the errors as appropriate.
	\item Finally, the {\sc Bib}\TeX\ database should be correct.
	\end{enumerate}
\item Information that I would include when citing common sources of information, such as {\it Wikipedia}, using the Harvard Referencing Style: \vspace{-0.3cm}
	\begin{enumerate} \itemsep -2pt
	\item Wikipedia contributors, ``TITLE\_OF\_THE\_ARTICLE,'' in \{$\backslash$it Wikipedia, The Free Encyclopedia: CATEGORY\}, Wikimedia Foundation, San Francisco, CA, MONTH DATE, YEAR. Available online at: $\backslash$url\{URL\}; last accessed on August 26, 2014.
	\item Wikibooks contributors, ``CHAPTER\_NAME,'' in \{$\backslash$it TITLE\_OF\_THE\_BOOK\}, Wikibooks: Open books for an open world, Wikimedia Foundation, San Francisco, CA, MONTH DATE, YEAR. Available online at: $\backslash$url\{URL\}; last accessed on August 26, 2014.
	\item Wikibooks contributors, ``SECTION,'' in \{$\backslash$it CHAPTER\} of \{$\backslash$it TITLE OF THE BOOK\}, Wikibooks: Open books for an open world, Wikimedia Foundation, San Francisco, CA, MONTH DATE, YEAR. Available online at: $\backslash$url\{URL\}; last accessed on August 26, 2014.
	\item Wikibooks contributors, ``TITLE\_OF\_THE\_BOOK,'' Wikibooks: Open books for an open world, Wikimedia Foundation, San Francisco, CA, MONTH DATE, YEAR. Available online at: $\backslash$url\{URL\}; last accessed on August 26, 2014.
	\item Wikiquote contributors, ``TITLE,'' Wikiquote, Wikimedia Foundation, San Francisco, CA, MONTH DATE, YEAR. Available online at: $\backslash$url\{URL\}; last accessed on August 26, 2014.
	\item Wiktionary contributors, ``TITLE,'' Wiktionary, Wikimedia Foundation, San Francisco, CA, MONTH DATE, YEAR. Available online at: $\backslash$url\{URL\}; last accessed on August 26, 2014.
	\item Dictionary.com, ``WORD,'' IAC, Oakland, CA, MONTH DATE, YEAR. Available online at: $\backslash$url\{URL\}; last accessed on August 26, 2014.
	\item AUTHOR, ``TITLE,'' in \{$\backslash$it The New York Times: The Opinion Pages: Op-Ed Contributor\}, The New York Times Company, New York, NY, MONTH DATE, YEAR. Available online at: $\backslash$url\{URL\}; last accessed on August 26, 2014.
	\item AUTHOR, ``QUESTION'', in \{$\backslash$it CATEGORY\}, Quora, Inc., Mountain View, CA, MONTH DATE, YEAR. Available online at: $\backslash$url\{URL\}; last accessed on August 26, 2014.
	\item AUTHOR, Answer to ``QUESTION'', in \{$\backslash$it CATEGORY: QUESTION\}, Quora, Inc., Mountain View, CA, MONTH DATE, YEAR. Available online at: $\backslash$url\{URL\}; last accessed on August 26, 2014.
	\item AUTHOR, ``TITLE OF POST'', in \{$\backslash$it BLOG TITLE\}, Quora, Inc., Mountain View, CA, MONTH DATE, YEAR. Available online at: $\backslash$url\{URL\}; last accessed on August 26, 2014.
	\item AUTHOR, ``TITLE,'' Stack Exchange Inc., New York, NY, MONTH DAY, YEAR. Available online from \{$\backslash$it Stack Exchange Inc.: Stack Overflow: Questions\} at: $\backslash$url\{URL\}; March 16, 2016 was the last accessed date.
	\item AUTHOR, ``TITLE OF REPOSITORY,'' GitHub, Inc., San Francisco, CA, MONTH DAY, YEAR. Available online from \{$\backslash$it \{GitHub: GitHub USERNAME (or NAME OF ORGANIZATION)\}: at: $\backslash$url\{URL\}; March 16, 2016 was the last accessed date.
	\item AUTHOR, ``TITLE OF PAPER,'' Cornell University, Ithaca, NY, MONTH DAY, YEAR. Available online (as Version XYZ) from \{$\backslash$it \{arXiv: FIELD(s)\}: at: $\backslash$url\{URL\}; March 16, 2016 was the last accessed date.
	\item When {\sc Bib}\TeX\ entries are created for the aforementioned sources of information, populate the appropriate fields so that each information in the aforementioned sources are included in the {\sc Bib}\TeX\ entries.
	\item For other organizations, communities, and groups, use the term ``contributors'' instead of ``members,'' unless otherwise specified.
	\end{enumerate}
\item Refer to the file ``bibtex-template.txt'' for templates for selected {\sc Bib}\TeX\ entry types. The more information that you can put in, the easier you can protect yourself from accusations of plagiarism and to make it easier for people (including yourself) to find the reference again. This is especially true for Web-based references/resources.
\item When the names of authors and editors are unknown, list them as ``Anonymous contributors.''
\end{enumerate}



%%%%%%%%%%%%%%%%%%%%%%%%%%%%%%%%%%%%%%%%%%%
\subsection{Recommended Fields for {\sc Bib}\TeX\ Entries}
\label{ssec:RecommendedFieldsforBibTeXEntries}


The recommended fields for {\sc Bib}\TeX\ entries are: \vspace{-0.3cm}
\begin{enumerate} \itemsep -4pt
\item booklet: \vspace{-0.3cm}
	\begin{enumerate} \itemsep -2pt
	\item Address
	\item Author or Editor
	\item Edition
	\item Howpublished
	\item Keywords
	\item Month
	\item Pages
	\item Publisher
	\item Series
	\item Title
	\item URL
	\item Volume
	\item Year
	\end{enumerate}
\item techreport: \vspace{-0.3cm}
	\begin{enumerate} \itemsep -2pt
	\item Address
	\item Author or Editor
	\item DOI
	\item Howpublished
	\item Institution
	\item Keywords
	\item Month
	\item Number
	\item Title
	\item URL
	\item Year
	\end{enumerate}
\item proceedings: \vspace{-0.3cm}
	\begin{enumerate} \itemsep -2pt
	\item Address
	\item DOI
	\item Editor
	\item Keywords
	\item Month
	\item Organization
	\item Publisher
	\item Series
	\item Title
	\item Volume
	\item Year
	\end{enumerate}
\item manual: \vspace{-0.3cm}
	\begin{enumerate} \itemsep -2pt
	\item Address
	\item Author or Editor
	\item Howpublished
	\item Keywords
	\item Month
	\item Organization
	\item Title
	\item URL
	\item Year
	\end{enumerate}
\item inbook: \vspace{-0.3cm}
	\begin{enumerate} \itemsep -2pt
	\item Address
	\item Author or Editor
	\item Booktitle
	\item Chapter (optional)
	\item DOI
	\item Edition
	\item Howpublished
	\item Keywords
	\item Number
	\item Pages
	\item Publisher
	\item Series
	\item Title
	\item Type
	\item URL
	\item Volume
	\item Year
	\end{enumerate}
\item incollection: \vspace{-0.3cm}
	\begin{enumerate} \itemsep -2pt
	\item Address
	\item Author or Editor
	\item Booktitle
	\item Chapter
	\item DOI
	\item Edition
	\item Howpublished
	\item Keywords
	\item Number
	\item Pages
	\item Publisher
	\item Series
	\item Title
	\item URL
	\item Volume
	\item Year
	\end{enumerate}
\item inproceedings: \vspace{-0.3cm}
	\begin{enumerate} \itemsep -2pt
	\item Address
	\item Author
	\item Booktitle
	\item DOI
	\item Keywords
	\item Month
	\item Number
	\item Organization
	\item Pages
	\item Publisher
	\item Series
	\item Title
	\item Volume
	\item Year
	\end{enumerate}
\item article: \vspace{-0.3cm}
	\begin{enumerate} \itemsep -2pt
	\item Address
	\item Author
	\item DOI
	\item Journal
	\item Keywords
	\item Month
	\item Number
	\item Pages
	\item Publisher
	\item Title
	\item Volume
	\item Year
	\end{enumerate}
\item phdthesis (or mastersthesis): \vspace{-0.3cm}
	\begin{enumerate} \itemsep -2pt
	\item Address
	\item Author
	\item DOI (there are multiple research universities that assign DOIs to Ph.D. dissertations, and possibly Masters theses)
	\item Howpublished
	\item Keywords
	\item Month
	\item Number: \vspace{-0.2cm}
		\begin{enumerate} \itemsep -2pt
		\item Some {\sc Bib}\TeX\ bibliography styles don't indicate the number, or show the number but hide some other important information.
		\item Hence, if the number is important include it in the Annote field.
		\end{enumerate}
	\item School
	\item Title
	\item URL
	\item Year
	\item Annote
	\end{enumerate}
\item misc: \vspace{-0.3cm}
	\begin{enumerate} \itemsep -2pt
	\item Address
	\item Author
	\item Howpublished
	\item Keywords
	\item Month
	\item Publisher or School
	\item Title
	\item URL
	\item Year
	\end{enumerate}
\item book: \vspace{-0.3cm}
	\begin{enumerate} \itemsep -2pt
	\item Address
	\item Author
	\item DOI
	\item Edition
	\item Keywords
	\item Month
	\item Pages
	\item Publisher
	\item Series
	\item Title
	\item Volume
	\item Year
	\end{enumerate}
\end{enumerate}


For additional notes and annotations for publications, I can use the {\sc Bib}\TeX\ fields {\tt Annote} or {\tt Note} to include that information in the {\sc Bib}\TeX\ entry. Please kindly note that information from the {\sc Bib}\TeX\ field {\tt Note} may end up in the reference list (or list of references), while information from the {\sc Bib}\TeX\ field {\tt Annote} would not. \\

In addition, to cite the specific page numbers of interest, use the {\sc Bib}\TeX\ field {\tt Pages}.




Ebury Digital






%%%%%%%%%%%%%%%%%%%%%%%%%%%%%%%%%%%%%%%%%%%
\subsection{Template Information for the {\tt Annote} and {\tt Howpublished} {\sc Bib}\TeX\ Fields}
\label{ssec:TemplateInformationForTheAnnoteBibTeXField}

This is a list of template information for the {\tt Annote} {\sc Bib}\TeX\ field to help me identify particular information about the document/publication: \vspace{-0.3cm}
\begin{enumerate} \itemsep -4pt
\item Alternate names of authors: blah, blah blah, blah blah blah, \dots \vspace{-0.3cm}
	\begin{enumerate} \itemsep -2pt
	\item Deprecated: \vspace{-0.2cm}
		\begin{enumerate} \itemsep -2pt
		\item Authors' names (alternate): BLAH-1 and BLAH-2.
		\item Full names of authors: BLAH-1 and BLAH-2.
		\item Alternate names of $[$Author XYZ$]$: BLAH-1 and BLAH-2.
		\item Actual name of author: BLAH.
		\item Actual name of co-authors: BLAH and BLAH.
		\item No authors are associated with this document, but the authors associated with the BibTeX-KEY document are also associated with this document; they are published as a set. Hence, instead of using anonymous contributors, I use their names instead.
		\end{enumerate}
	\item Previously indicated the staff or members of the following organizations as co-authors, instead of individuals: ORG-ABC, ORG-DEF, and ORG HIG.
	\item Previously indicated set of co-authors: Blah-1, Blah-2, Blah-3, and Blah-4.
	\item For publications that many contributors, select the major contributors to have their names listed and include the following to complete the list of co-authors: ``and other contributors'', or ``and other $[$PROJECT-NAME$]$ contributors''.
	\end{enumerate}
\item Editors: BLAH ... BLAH ... BLAH.: \vspace{-0.3cm}
	\begin{enumerate} \itemsep -2pt
	\item When citing from this entry, shift the names of the authors from the author field to the editor field. \vspace{-0.2cm}
		\begin{enumerate} \itemsep -2pt
		\item Since certain {\sc Bib}\TeX\ styles require a non-empty author field, I have placed the information of editors in the authors field. 
		\item Swap them back when using them for publications, by changing the {\tt Author} field to the {\tt Editor} field.
		%	For a lot, if not most of these publications, I have included the aforementioned statement to indicate this. I would debug this as I come across them.
		\end{enumerate}
	\item Editors of the book: BLAH-1 and BLAH-2.
	\item Editors of conference proceedings: BLAH-1, BLAH-2, and BLAH-3.
	\end{enumerate}
\item Publication issues: \vspace{-0.3cm}
	\begin{enumerate} \itemsep -2pt
	\item Repeated/duplicate {\sc Bib}\TeX\ entries: \vspace{-0.2cm}
		\begin{enumerate} \itemsep -2pt
		%	Repeat entry. See \cite{BibTeX-KEYS}.
		\item Repeat entry. See $\backslash$cite\{BIbTeX keys\}.
		\item Merged with the entry for BLAH, since they refer to the same publication.
		\end{enumerate}
	\item Dates of publication: \vspace{-0.2cm}
		\begin{enumerate} \itemsep -2pt
		\item Originally published in: 20XY. See BIBTEX-KEY.
		\item Also, published in: 20XY. See BIBTEX-KEY.
		\item Conference proceedings published in: 20XY.
		\item Reprinted in 20XY by BLAH. See BIBTEX-KEY.
		\item Copyright renewed in 20XY.
		\item Years of publication for previous editions: First edition in 19XY-1, Second edition in 200X-1, Third edition in 201X-1, and Fourth edition in 202X-1.
		\item Previously indicated dates are probably incorrect: DATE-1, DATE-2, and DATE-3.
		\end{enumerate}
	\item Addresses of publisher(s): \vspace{-0.2cm}
		\begin{enumerate} \itemsep -2pt
		\item Address of publisher, rather than conference location is: BLAH.
		\item Address is officially stated as: BLAH.
		\item Address of conference venue is previous indicated as: BLAH.: \vspace{-0.1cm}
			\begin{enumerate} \itemsep -1pt
			\item Address of conference venue is previous indicated as: Muenchen, Germany. It is a German name (or rather, M{\"{u}}nchen, Germany) of Munich, Germany.
			\end{enumerate}
		\item Deprecated templates: \vspace{-0.1cm}
			\begin{enumerate} \itemsep -1pt
			\item Address is officially stated in: BLAH.
			\end{enumerate}
		\end{enumerate}
	\item Alternate {\bf names} of publisher: \vspace{-0.2cm}
		\begin{enumerate} \itemsep -2pt
		\item Alternate publisher name: BLAH.
		\item Unknown publisher; see the following for an alternate BibTeX entry with an associated publisher: BIBTEX-KEY.
		\item Unknown publisher; see the following for an alternate BibTeX entry without an associated publisher: BIBTEX-KEY.
		\end{enumerate}
	\item Organizations associated with the publication: \vspace{-0.2cm}
		\begin{enumerate} \itemsep -2pt
		\item Organizations that are recognized as involved in this report: BLAH and BLAH.
		\item Deprecated templates: \vspace{-0.1cm}
			\begin{enumerate} \itemsep -1pt
			\item BLAH and BLAH are recognized as organizations involved in this report.
			\end{enumerate}
		\end{enumerate}
	\item Alternate publication {\bf titles} or {\bf series} (and alternate {\bf volume number}): \vspace{-0.2cm}
		\begin{enumerate} \itemsep -2pt
		\item Also, published as BLAH (old series title and old volume number).
		\item Alternate series title: BLAH.
		\item Regarding translations, and translated titles in a particular language, see the set of items for {\it {\bf Translation}-based templates}.
		\item This title was used in the reprint in 20XY.
		\end{enumerate}
	\item {\bf Other publishers} of the publication: \vspace{-0.2cm}
		\begin{enumerate} \itemsep -2pt
		\item Previously indicated publisher is: BLAH.
		\item Additional publisher: BLAH.
		\item Distributed by: BLAH.
		\end{enumerate}
	\item Alternate titles of the publication: \vspace{-0.2cm}
		\begin{enumerate} \itemsep -2pt
		\item Alternate title: {\it Another title for the document/publication}.
		\item Alternate subtitle: {\it BLAH}.
		\item Subsubtitle: {\it BLAH}.
		\item Alternate subsubtitle: {\it BLAH}.
		\item Differences from the online version with the print version: \vspace{-0.1cm}
			\begin{enumerate} \itemsep -1pt
			\item Differences from the online version with the print version. Title of the print version: ``How My Generation Broke America'' (on the cover of the magazine). Title of the online version: ``My Generation was Supposed to Level America's Playing Field. Instead, We Rigged It for Ourselves'' (online article).
			\item Differences from the online version with the print version. Title of the print version on the cover of the magazine: How My Generation Broke America. Title of the print version for the actual article: My Generation was Supposed to Level America's Playing Field. Instead, We Rigged It for Ourselves.
			\end{enumerate}
		\item Differences from the front cover version with the title page version (in the front matter): \vspace{-0.1cm}
			\begin{enumerate} \itemsep -1pt
			\item Differences from the front cover version with the title page version (in the front matter). Title of the front cover version: ``Beautiful Souls: The Courage and Conscience of Ordinary People in Extraordinary Times'' (on the front cover version). Title of the title page version (in the front matter): ``Beautiful Souls: Saying No, Breaking Ranks, and Heeding the Voice of Conscience in Dark Times'' (title page version).
			\end{enumerate}
		\end{enumerate}
	\item Alternate series of publication: \vspace{-0.2cm}
		\begin{enumerate} \itemsep -2pt
		\item Possible series: BLAH.
		\item Series was previously known as: BLAH.
		\end{enumerate}
	\item Alternate editions of publication: \vspace{-0.2cm}
		\begin{enumerate} \itemsep -2pt
		\item Reprinted, paperback edition.
		\item Previous editions are published by BLAH in: 1987, 1991, 1999, and 2009.
		\end{enumerate}
	\item DOI problems: \vspace{-0.2cm}
		\begin{enumerate} \itemsep -2pt
		\item Warning: DOI does not link back to the Web page {\tt URL}.
		\item Digital Object Identifier (DOI) redirects to the Web page for the BLAH-X edition of the book, while the PDF preview document on that Web page is for the BLAH-Y edition of the book.
		\end{enumerate}
	\item URLs: \vspace{-0.2cm}
		\begin{enumerate} \itemsep -2pt
		\item Alternate {\tt URL}: {\tt BLAH}.
		\item Available online as \{$\backslash$it The $\backslash$LaTeX$\backslash$ Project\} at: $\backslash$url\{\}.
		\item Scans of the magazines are available at: {\tt URL}.
		\item PDF copy of a draft of the publication is available at: URL.
		\end{enumerate}
	\item Subsets of publications: \vspace{-0.2cm}
		\begin{enumerate} \itemsep -2pt
		\item For the following {\sc Bib}\TeX\ entry types: \vspace{-0.1cm}
			\begin{enumerate} \itemsep -1pt
			\item {\tt incollection} (with titles for chapters, sections, subsections, and subsubsections), for the {\tt book}, {\tt booklet}, {\tt techreport}, {\tt phdthesis}, and {\tt mastersthesis}, and {\tt manual} {\sc Bib}\TeX\ entry types.
			\item {\tt inbook} (without titles for chapters, sections, subsections, and subsubsections), for the {\tt book}, {\tt booklet}, {\tt techreport}, {\tt phdthesis}, and {\tt mastersthesis}, and {\tt manual} {\sc Bib}\TeX\ entry types.
			\item {\tt inproceedings}, for the {\tt proceedings} {\sc Bib}\TeX\ entry type.
			\item {\tt article}, for the {\tt book} and {\tt booklet} {\sc Bib}\TeX\ entry types.
			\end{enumerate}
		\item Part of the book/report: BLAH.
		\item This article is part of the journal/magazine issue: BibTeX-KEY.
		\end{enumerate}
	\item Abstracts of publications listed as separate publications: \vspace{-0.2cm}
		\begin{enumerate} \itemsep -2pt
		\item Abstract is available from BLAH at: BLAH-BLAH-BLAH.
		\end{enumerate}
	\item Multiple fields of publication: \vspace{-0.2cm}
		\begin{enumerate} \itemsep -2pt
		\item Originally published by BLAH, ADDRESS, in YEAR.
		\item Originally published by BLAH, ADDRESS.
		\item Also, published by BLAH, ADDRESS, in YEAR.
		\item Also, published by BLAH, ADDRESS.
		\item Deprecated templates: \vspace{-0.1cm}
			\begin{enumerate} \itemsep -1pt
			\item Originally published by BLAH in BLAH--BLAH.
			\item Also, published by BLAH in BLAH--BLAH.
			\end{enumerate}
		\end{enumerate}
	\end{enumerate}
\item Templates for the {\tt Howpublished} field: \vspace{-0.3cm}
	\begin{enumerate} \itemsep -2pt
	\item Received a copy of this {\it report/document/publication/thesis/dissertation} by email.
	\item Received a copy of this report by email.
	\end{enumerate}
\item {\bf Translation}-based templates: \vspace{-0.3cm}
	\begin{enumerate} \itemsep -2pt
	\item Translated by: BLAH.
	\item BibTeX key of English translation: BLAH.
	\item Original French title: BLAH.
	\item Translated English title: BLAH.
	\item Deprecated templates: \vspace{-0.2cm}
		\begin{enumerate} \itemsep -2pt
		\item English translation of title: BLAH.
		\end{enumerate}
	\end{enumerate}
\item Associated files for the publication: \vspace{-0.3cm}
	\begin{enumerate} \itemsep -2pt
	\item Associated files are formerly named: {\tt BLAH.file\_extension}, ...
	\item Refer to BLAH-BibTeX-Key for a PDF copy of this publication.
	\item Refer to BLAH-BibTeX-Key for a PDF copy of this publication. Only the front matter is available in the PDF copy.
	\end{enumerate}
\item Tracking updates to {\sc Bib}\TeX\ keys: \vspace{-0.3cm}
	\begin{enumerate} \itemsep -2pt
	\item Former BibTeX key is: {\tt Former BibTeX key}.
	\item Former BibTeX keys are: {\tt Former BibTeX key \#1} and {\tt Former BibTeX key \#2} \dots
	\item Keep BibTeX key as: BIPMJCGMWG2VIMcontributors2012. This enables the acronym to be differentiated from the term "contributors".
	\end{enumerate}
\item Similar publications: \vspace{-0.3cm}
	\begin{enumerate} \itemsep -2pt
	\item Similar dissertations: $[${\tt List of {\rm {\sc Bib}\TeX}\ keys}$]$.
	\item Similar publications: $[${\tt List of {\rm {\sc Bib}\TeX}\ keys}$]$.
	\item Differences between multiple publications, which are similar: \vspace{-0.2cm}
		\begin{enumerate} \itemsep -2pt
		\item Differences between multiple publications are listed as follows. PUBLICATION1 addresses XYZ, but PUBLICATION2 and PUBLICATION3 do not.
		\end{enumerate}
	\end{enumerate}
\item {\tt Howpublished} {\sc Bib}\TeX\ field components: \vspace{-0.3cm}
	\begin{enumerate} \itemsep -2pt
	\item the last accessed date is unknown
	\item BLAH-1 and BLAH-2 were the last accessed dates
	\end{enumerate}
\item Questionable correctness of {\sc Bib}\TeX\ entries: \vspace{-0.3cm}
	\begin{enumerate} \itemsep -2pt
	\item Questionable correctness of BibTeX entry. Information is probably obtained from secondary sources of information, such as publications that cite this publication.
	\end{enumerate}.
\end{enumerate}







%These work:
%+ \AmS-\LaTeX.



%	The following do not work.
%$\varepsilon \kappa {\'{\alpha}} \beta \eta$
%$\varepsilon\kappa{\'{\alpha}}\beta\eta$
%$\varepsilon\kappa\alpha\beta\eta$ \\
%\varepsilon\kappa\alpha\beta\eta
%$\varepsilon\kappa${\|{$\alpha$}}$\beta\eta$
%{\'{a}}{\'{$\alpha$}}
%{\'{a}}${\'{\alpha}}$
%{\'{a}}$\'\alpha$
%	The following works.
%$\varepsilon\kappa\alpha\beta\eta$ \\
%{\'{a}}$\acute{\alpha}$








%%%%%%%%%%%%%%%%%%%%%%%%%%%%%%%%%%%%%%%%%%%
\subsection{Template Information for the {\tt Howpublished} {\sc Bib}\TeX\ Field}
\label{ssec:TemplateInformationForTheHowpublishedBibTeXField}

For the {\tt Howpublished} {\sc Bib}\TeX\ field, the suggested template information is: \vspace{-0.3cm}
\begin{itemize} \itemsep -4pt
\item Available online at: $\backslash$url\{\}; self-published; MONTH DAY, YEAR was the last accessed date.
\item Available online from {\it main Web page: ABC section: XYZ subsection} as Version X.Y.Z at: $\backslash$url\{\}; self-published; MONTH DAY, YEAR was the last accessed date.
\item Available online from {\it main Web page: ABC section: XYZ subsection} in Italian at: $\backslash$url\{\}; self-published; MONTH DAY, YEAR was the last accessed date.
\item Available online from {\it main Web page: ABC section: XYZ subsection} in Italian and Spanish at: $\backslash$url\{\}; self-published; MONTH DAY, YEAR was the last accessed date.
\end{itemize}

When the date is not known so that it can be provided in the ``MONTH DAY, YEAR'' format, use the following phrase instead of ``MONTH DAY, YEAR was the last accessed date''. \vspace{-0.3cm}
\begin{itemize} \itemsep -4pt
\item the last accessed date is unknown
\end{itemize}










%%%%%%%%%%%%%%%%%%%%%%%%%%%%%%%%%%%%%%%%%%%
\subsection{Additional Recommendations for Managing a Shared {\sc Bib}\TeX\ Database}
\label{ssec:AdditionalRecommendationsForManagingASharedBibTeXDatabase}



Use {\tt crossref} for {\sc Bib}\TeX\ entries using the {\sc Bib}\TeX\ entry type {\tt inproceedings} to share information about the conference proceedings without having to copy and paste fields that are commonly shared by articles/papers in conference proceedings \cite[\S12.2.3, pp. 234]{Kopka2004}. It might not work with {\sc Bib}\TeX\ entry types {\tt inbook} and {\tt incollection} for parts, subsections, sections, and chapters of books; this is not mentioned in {\sc Bib}\TeX\ specifications/references, and {\tt crossref} is probably not supported for these {\sc Bib}\TeX\ entry types. Hence, I am currently not using {\tt crossref}, unless I have to cite multiple (more than 15, $>15$, or even five, $>5$) conference papers in a given conference proceedings. \\



When copying (and pasting) text from a document or a Web page, non-ASCII hidden characters may be accidentally copied from the source (document or Web page) to the {\sc Bib}\TeX\ database. Consequently, this can cause command-line utilities for UNIX-like operating systems, such as {\tt grep}, to fail to recognize the {\sc Bib}\TeX\ database as a text file. While some text editors or integrated development environments (IDEs) have features that highlight or indicate such non-ASCII hidden characters, they are not effective for finding/detecting these characters in large text files ({\sc Bib}\TeX\ databases in this case) without knowing what these characters are. If these characters are known, the search feature of these text editors or IDEs can find and delete these non-ASCII hidden characters. \\

An alternative is to develop and use a script to concatenate all the {\sc Bib}\TeX\ keys (of a {\sc Bib}\TeX\ database) in a {\tt $\backslash$cite\{\}} \LaTeX\ command to automatically generate a listing of the references in a particular {\sc Bib}\TeX\ style of your choice. This results in the \LaTeX\ and {\sc Bib}\TeX\ interpreters parsing and processing your \LaTeX\ and {\sc Bib}\TeX\ sources. If non-ASCII hidden characters exist in your \LaTeX\ and {\sc Bib}\TeX\ source files, this process should abruptly pause your \LaTeX\ and {\sc Bib}\TeX\ interpretation processes to warn you of the syntax errors due to these characters. \\

In addition, note that corruption in memory subsystems or storage devices (such as a solid-state drives) and data transfer between computers can result in corrupting text files to produce non-ASCII hidden characters in these files.





%%%%%%%%%%%%%%%%%%%%%%%%%%%%%%%%%%%%%%%%%%%
\section{Coding Standard}
\label{sec:CodingStandard}

This is a guideline for {\tt Doxygen}-supported \cite{vanHeesch2016}, {\tt Javadoc}-based \cite{Long1995} coding standard that shall be used for this boilerplate code project and other projects. The term ``coding standard'' is used interchangeably/synonymously with coding style, coding style guide, coding guideline, coding scheme, code convention, code documentation guideline, programming guideline, or programming style. Our coding style/standard shall be self-documenting. The documentation generator that shall be supported is: {\tt Doxygen}. Since we are using {\tt Doxygen} for generating documentation, we can use \LaTeX\ to provide richer markup. \\

{\bf Document the known bugs for each function/method.} \\

%	My indent style would be the {\it 1TBS} variant of the {\it K{\rm \&}R} style \cite{WikipediaContributors2016j}, which is an abbreviation of ``{\it The One True Brace Style}'' \cite{WikipediaContributors2016j}. It is also equivalent to the {\it Kernel Normal Form style} (or {\it BSD KNF style}) \cite{WikipediaContributors2016j}. \\
Our indent style would be the {\it 1TBS} variant of the {\it K{\rm \&}R} style, which is an abbreviation of ``{\it The One True Brace Style}''. It is also equivalent to the {\it Kernel Normal Form style} (or {\it BSD KNF style}) \cite{WikipediaContributors2016j}. \\

Classes, functions/methods, constants, macros, and static and instance variables shall be named using complete words or well-known abbreviations that are concatenated with an underscore in {\it C++}; this is a deviation from the {\it Hungarian notation} that uses an upper case letter to distinguish words/abbreviations in the name (i.e., the {\it Start case style of writing}; see letter case). That is, the naming convention followed is using multiple-word identifiers, via delimiter-separated words rather than letter-case separated words (e.g., {\it Hungarian notation}) \cite{WikipediaContributors2017}. \\

For {\it C++} programs, the following tags shall be used in the comments: \vspace{-0.3cm}
\begin{enumerate} \itemsep -4pt
\item @author {\it Author's\_Name}: indicate the author ({\it Author's\_Name}) of the file/function \vspace{-0.3cm}
	\begin{enumerate} \itemsep -2pt
	\item @modified by NAME, DATE in ``Month Day, Year'' format.
	\end{enumerate}
\item @version {\it X.Y}: indicate the version ({\it X.Y}) of the file
\item @section {\it SECTION\_NAME}: indicate the section ({\it SECTION\_NAME}) of the file, which can be: {\it LICENSE} or {\it DESCRIPTION}
\item @param {\it x}: indicate the parameter ({\it x}) of the constructor or function
\item @exception {\it Exception\_Name}, or @throws {\it Exception\_Name}: an exception that a function/method can throw
\item @return {\it Return\_Statement}: indicate the return (type and) action of the function
\item @see {\it reference}: a link to another element in the documentation; e.g., @see {\it Class\_Name}, or @see {\it Class\_Name\#member\_function\_name}
\item @since {\it X.Y: Month-Day-Year}: This functionality has been added since version {\it X.Y} (and on the date {\it Month-Day-Year})
\item @deprecated {\it description}: Describe an outdated function/method, and indicate when the function/method has deprecated
\item ``@link {\it ... URL...} @endlink'' is used to include hyperlinks in the generated documentation for Doxygen
\item \#\#\#\# IMPORTANT NOTES: Notes that are critical for helping the reader understanding assumptions and decisions made while developing the software
\item @todo($<$message$>$, $<$version$>$) (or \#\#\#\# TO BE COMPLETED): Task to be finished at a later time. If it is busywork (or, busy work), indicate that it is busywork.
%	#### TO BE FIXED
\item \#\#\#\# TO BE FIXED: Task to be fixed at a later time, including: \vspace{-0.3cm}
	\begin{enumerate} \itemsep -2pt
	\item bugs to be debugged
	\item errors/faults to be fixed
	\item software/hardware/system architecture or source code to be refactored
	\item completion of feature implementation
	\end{enumerate}
\item @migration($<$message$>$, $<$version$>$): Code is being migrated to another function/method, or class.
\item See \url{http://www.stack.nl/~dimitri/doxygen/commands.html} for more information of tags that are recognized by {\tt Doxygen}.
\item @pre (or @precondition): Precondition(s) of the function.
\item @assert (or @assertion): Assertion(s) of the function.
\item @post (or @postcondition): Postcondition(s) of the function.
\end{enumerate}

The order of tags in different sections of the {\it C++} code is given as follows: \vspace{-0.3cm}
\begin{enumerate} \itemsep -4pt
\item Headers/Interfaces and Classes: @version, @author, @since, @link, @todo, @deprecated, @migration, and @see
\item Constructors: @param, @throws, @since, @link, @todo, @deprecated, @migration, and @see. For collaborators modifying or extending my code, they should include the @version and @author tags before the @param tag(s).
\item Functions/Methods: @param, @pre, @assert, @post, @return, @throws, @since, @link, @todo, @deprecated, @migration, and @see. For collaborators modifying or extending my code, they should include the @version and @author tags before the @param tag(s).
\item Variables can use the @see tags.
\item The @deprecated tag can be used for headers/interfaces, classes, constructors, functions/methods, and variables.
\end{enumerate}

Additional coding style guidelines can be found in \cite{Costan2019,Basalaj2013,ProgrammingResearchLtdStaff2013,Thereska20XY,Hoff2008a,ScienceInfusionSoftwareEngineeringProcessGroupStaff2006,Cargill1992,Misfeldt2004}. \\

For a suggested coding style for {\it Python} and {\it Ruby} scripts, see \cite{Osborne2018b,vanRossum2013} and \cite{Macdonald20XY}, respectively. Regarding coding style guidelines for embedded {\it C}, see \cite{Barr2013,Labrosse1999}. In addition, there exists coding style guidelines for {\it Java} \cite{OracleCorporationStaff20XYa,Smith2003,OracleCorporationStaff2016b,OracleCorporationStaff2016,OracleCorporationStaff20XY,Long1995,Carrano2012} and {\it LabVIEW} \cite{Blume2007a,Conway2003}. Coding style guidelines for {\it Verilog} can be found at: \cite{Bening2001,Bening2000}. Likewise, the coding style guide for {\it SystemVerilog} can be found at \cite{Mintz2007}. For other coding style guidelines, see \cite{Wolf20XY,Cady2017,IntelCorporationStaff2015,Schneider2013a,Weatherwax2008,Laplante2012,McConnell2004,Feathers2005,Koopman2010,Valvano2007,Fingeroff2010,Shore2008,Schach2007,Kernighan1982}. {\it Google} style guides \cite{GoogleStyleguideContributors2019} has provided documentation about best practices \cite{Osborne2018a} for coding standards and the philosophy \cite{Osborne2018} of Google's coding standards. \\

%	See Section \S\ref{sssec:edasoftwaredevelopment} of my personal notes.
% code conventions, code documentation guidelines, coding guidelines, coding scheme, coding standards, coding style guide, coding styles, programming guidelines, and programming style
%	For my BibTeX database, use: coding scheme, coding standards, coding styles, coding guidelines, and programming style



While well-documented source code is desired, natural language programming \cite{WikipediaContributors2016i} is usually infeasible due to the choices of programming/computer languages used. Also, while literate programming \cite{Knuth1984,Knuth1992a,McConnell2004,Subramaniam2006,Schach2007,Oram2007,MullerHannemann2010}
 is encouraged, we are currently not following it due to the tedious process of developing software using literate programming. Hence, a short development time for well-commented, functionally correct, and efficient source code is prioritized over code written according to the literate programming approach.


%%%%%%%%%%%%%%%%%%%%%%%%%%%%%%%%%%%%%%%%%%%
\section{Exception Safety}
\label{sec:ExceptionSafety}

When developing software using programming/scripting languages that enable exceptions or errors to be thrown and caught, adopt "a set of contractual guidelines" \cite{WikipediaContributors2016f} to support exception/error management. This ``set of contractual guidelines'' is based on exception safety guarantees in {\it C++} \cite{Abrahams1998,Abrahams2001,WikipediaContributors2016f} \cite[Subsection \S4.4 on ``Writing exception safe code'']{WikibooksContributors2016}. \\

%\cite{Abrahams1998,Abrahams2001,WikibooksContributors2016,WikipediaContributors2016f}

The levels of exception/error safety listed in descending order of safety guarantees are \cite{Abrahams1998,Abrahams2001,WikibooksContributors2016,WikipediaContributors2016f}: \vspace{-0.3cm}
\begin{enumerate} \itemsep -4pt
\item no throw guarantee, or failure transparency: ``Best level of exception safety.''
\item strong exception safety, commit/rollback semantics, or no-change guarantee
\item basic exception safety
\item minimal exception safety, no-leak guarantee
\item no exception safety: ``No guarantees are made. (Worst level of exception safety)''
\end{enumerate}

These aforementioned levels of exception/error safety can be partially handled. Also, the use of guards is strongly recommended for making the software and library (or, circuit or system) exception safe. \\

These guidelines about exceptions help software developers know what to do about fatal exceptions, boneheaded exceptions, vexing exceptions (due to unfortunate design decisions). Vexing exceptions and boneheaded exceptions, to a lesser extent, are preventable exceptions \cite{Lippert2008}. Hence, we should develop software that avoids triggering preventable exceptions. \\

Please judiciously consider what to do with the semipredicate problem \cite{WikipediaContributors2016e}.



%%%%%%%%%%%%%%%%%%%%%%%%%%%%%%%%%%%%%%%%%%%
\section{Suggested Software Architecture}
\label{sec:SuggestedSoftwareArchitecture}

At the software system level, the software architecture can be described by the following modules/components: \vspace{-0.3cm}
\begin{enumerate} \itemsep -4pt
\item parser(s): \vspace{-0.3cm}
	\begin{enumerate} \itemsep -2pt
	\item For input benchmarks
	\end{enumerate}
\item utilities: \vspace{-0.3cm}
	\begin{enumerate} \itemsep -2pt
	\item output generator(s)
	\item flag/switch -based printing information to standard output/error: \vspace{-0.2cm}
		\begin{enumerate} \itemsep -2pt
		\item Print statements only when debugging mode is on.
		\item Else, squelch print/trace statements to speed up computation/performance.
		\end{enumerate}
	\end{enumerate}
\item solvers: \vspace{-0.3cm}
	\begin{enumerate} \itemsep -2pt
	\item ODE solver(s) for ordinary differential equations (ODEs): \vspace{-0.2cm}
		\begin{enumerate} \itemsep -2pt
		\item ODE solver(s) for nonlinear ODEs.
		\end{enumerate}
	\item PDE solver(s) for partial differential equations (PDEs): \vspace{-0.2cm}
		\begin{enumerate} \itemsep -2pt
		\item PDE solver(s) for nonlinear PDEs.
		\end{enumerate}
	\item satisfiability modulo theories (SMT) solver(s)
	\item boolean/proposition satisfiability (SAT) solver(s)
	\item maximum satisfiability modulo theories (Max-SMT) solver(s)
	\item maximum satisfiability (Max-SAT) solver(s)
	\item pseudo-boolean optimization (PBO) solver(s)
	\item quadratic unconstrained binary optimization (QUBO) solver(s)
	\item weighted boolean optimization (WBO) solver(s)
	\item framework for algorithmic portfolio optimization
	\end{enumerate}
\item data structures: \vspace{-0.3cm}
	\begin{enumerate} \itemsep -2pt
	\item directed graphs: \vspace{-0.2cm}
		\begin{enumerate} \itemsep -2pt
		\item directed acyclic graphs (DAGs)
		\item binary decision diagrams (BDDs)
		\item AND-inverter graphs (AIGs)
		\end{enumerate}
	\item undirected graphs: \vspace{-0.2cm}
		\begin{enumerate} \itemsep -2pt
		\item heaps
		\item trees
		\end{enumerate}
	\item maps, dictionaries, and hash tables
	\end{enumerate}
\item graphical user interface (GUI), if required.
\end{enumerate}




Lastly, suggestions are not available for digital and mixed-signal integrated circuits (ICs) and VLSI systems, such as system-on-chips (SoCs). More work needs to be done in terms of looking at hardware refactoring, and hardware design patterns.




%%%%%%%%%%%%%%%%%%%%%%%%%%%%%%%%%%%%%%%%%%%
\section{Adoption of Best Practices}
\label{sec:AdoptionOfBestPractices}


%	References for ``reproducible workflow'' are also references for ``research reproducibility'' and ``reproducible research.''
%	version control system, revision control, revision control system, configuration management, software configuration management...
%	References for version/revision control with the design process of ICs, cyber-physical systems, and embedded systems: \cite{Chen2012,Stringham2010,Andrews2005,Pottie2005,Singh2004}
Where possible, we shall try to adopt multiple best practices from leading product teams (i.e., R\&D teams) in the semiconductor and IT industries, and also good researchers spanning electrical engineering and computer science. These practices include: research reproducibility and reproducible research \cite{Schiermeier2018,Baumer2017,CodeOceanstaff2017,DatopianAtomaticLtdIncStaff2017,Kim2017,Mailund2017,Barba2016,Gandrud2015,Liberman2015,Creswell2014,Gandrud2014,Stodden2014,RunMyCodeAssociationContributors2013,Blackburn20XY,Geier20XY,Krishnamurthi20XY}, build automation \cite{Hagelberg2017,Suereth2016,Keiter2013a,Collier2012,Crookshanks2012,Humble2011,Schach2011,Fingeroff2010,Riesbeck2009a,Schach2007,Raymond2004,Schach2002,Oram1991,Driscoll20XY}, distributed version/revision control (or software configuration management) \cite{AtlassianStaff2018b,Cady2017,Olson2016,Paarsch2016,Chacon2014,GitContributors2014,GitContributors2014a,GitHubStaff2014a,GitHubTrainingTeam2014,GitHubTrainingTeam2014a,GitHubTrainingTeam2014b,GitContributors2014b,Hamano2014,Fox2013,Wiegers2013,Collier2012,Crookshanks2012,Driscoll2012,Humble2011,Savage2011,Swicegood2010,Chacon2009,OSullivan2009,Shore2008,Albing2007,Masters2007,Oram2007,Schach2007,Raymond2004,Pressman2001,Capretz1996,McConnell1996}, design by contract (DbC; or, contract programming, programming by contract, or design-by-contract programming) \cite{Tarlinder2017,BrooksJr2010,Samek2009,Zeller2009,Huth2004,Stevens2000,Hailperin1999,Meyer1997a} when using the procedural/imperative programming paradigm \cite{WikipediaContributors2017e,Louden2012,Sestoft2012,Bailey2010a,Marrer2009,Lee2008c,Tucker2007,Watt2004,Mitchell2002,Pierce2002,Reynolds1998,BenAri1996} and Hoare logic \cite{Pierce2017,Zhan2017,Laplante2014,Kourie2012,Kundu2011,Samek2009,Zeller2009,Baldwin2004,Huth2004,Monin2003,Schwichtenberg2001,Misra2001a,Harel2000,Winskel1993}, regression testing \cite{Vucevic2012,Pugh2011,Schach2011,Agans2006,Schach2002}, automated software testing \cite{Cady2017,Pajankar2017b,Rother2017,Mili2015,Sale2014,CoverityStaff2012,Humble2011,Pugh2011,Rady2011,Schach2011,Turnquist2011,Laakmann2009,Whittaker2009,Zeller2009,Jalote2008,Dasso2007,Sommerville2007,Ong2006,Subramaniam2006,Mosley2002,Schach2002,Bentley2000,Stevens2000,Fewster1999,Kernighan1999}, and automated regression testing \cite{Rierson2013,Pugh2011}. An analogy of regression testing for software for VLSI design is regression verification \cite{Poulos2018,Poulos2018a,Adler2017,Adler2017a,Poulos2014,Poulos2014a,Poulos2013,Bombieri2012,Bailey2010a,Li2008a,Agans2006,Thoziyoor2005,Burgess2004,Singh2004,Glusman2003}. Similarly, endeavor to use the concepts of abstraction \cite{Lee2017a,Kropf1999,Hussein2008,Haynal2008a,Chauhan2007,Harris2007,Wang2006,Wang2004,Cong2003a,McIver2003,Stan2003,Furber2000,Kropf1999,Gil1998} and encapsulation \cite{Reddy2011,Marrer2009,Scott2009,Haynal2008a,Krogh2008a,Lee2008c,McIver2005,Baldwin2004,Raymond2004,Spolsky2004,Conway2003,Felleisen2001,Pratt2001,BrooksJr1995} with hierarchical design methodologies \cite{Buttazzo2011,Nuzzo2011,Fingeroff2010,Safarpour2009,Vaishnavi2008,Adya2004,Gupta2004,Krishnamachari1997,Krishnamachari1995}, hierarchical design space exploration \cite{Sun2011}, top-down hierarchical approach for design steps \cite{Kaeslin2015,Chen2012,Doboli2011,Gajski2009,Wieferink2008,Ho2007,Lu2007,SangiovanniVincentelli2007,Frevert2005} and bottom-up hierarchical approach for verification steps \cite{Chauhan2007,Wang2006,Frevert2005,McIver2003,Rashinkar2002}, or rather top-down approach of incremental verification (also known as the modified V approach) \cite{Bailey2010a}, and platform-based design \cite{Bailey2010a,Saponara2010,Gajski2009,Densmore2008,SangiovanniVincentelli2008,SangiovanniVincentelli2007,Burton2006,Saleh2006,Bailey2005,Madisetti2005,Martin2003,Chang2002,SangiovanniVincentelli2001,Keutzer2000} in our projects involving VLSI design. \\


%	hierarchical design, hierarchical design methodologies, hierarchical design space exploration, bottom-up verification approach, top-down design approach, top-down design flow, top-down design methodology, top-down hierarchical approach, top-down VLSI design, encapsulation, abstraction, hierarchical abstraction, levels of abstraction, verification across levels of abstraction, abstraction in hardware design, abstraction-based design methodology, abstraction-based verification
%	Hussein2008



%	Agile Manifesto, agile methodologies, agile methods, agile development
In addition, try to use agile (software development and VLSI design) methodologies \cite{Cady2017,Rothman2017,AltSimmons2016,Ghani2016,Rothman2016a,Jeffries2015,Jensen2015,Kelly2015,Olsen2015,Stellman2015,Babar2014,Goncalves2014,Laplante2014,Fox2013,Ambler2012,Collier2012,Crookshanks2012,Valacich2012,Appelo2011,Corr2011,Larsen2011a,Leffingwell2011,Plattner2011a,Pugh2011,Eckstein2010,Larman2010,Pressman2010,Rasmusson2010,Rothman2009,Stober2010,Davies2009,Larman2009,Martin2009a,DeLucia2008,Kelly2008,Shore2008,Younker2008,Cockburn2007,Martin2007,Rothman2007,Schach2007,37signalsStaff2006,Derby2006,Hunt2006a,Subramaniam2006,Beck2005,Coplien2005,Larman2005,Boehm2004,Larman2004,Ambler2003,Larman2003,Martin2002,Beedle2001,Beedle2001a} to develop software as well as design electronic circuits and systems \cite{Hennessy2018,Johnson2018a,McLellan2014a,Wilson2011a,Black2010a,Goering2010} \cite[Chapter 6, \S6.2.2.3, pp. 243]{Gerstlauer2001}, and cyber-physical systems (or embedded systems) \cite{MondragonTorres2013,MondragonTorres2011}. A strong motivation for using these methodologies and their associated practices is to reduce technical debt \cite{Tornhill2018,Brahma2017,Guo2016,Collier2012,Kruchten2012,Sterling2011,Rothman2007,37signalsStaff2006,Read2003,Myers1998}. \\

Also, carry out refactoring \cite{Fields2010,Fowler1999} on an ad-hoc basis to improve the software \cite{Dooley2017,Jeffries2015,Babar2014,Fox2013,Crookshanks2012,Bender2011,DiGennaro2011,Koopman2010,Shore2008,Spolsky2008,Meszaros2007,37signalsStaff2006,Feathers2005,Kerievsky2005,Evans2004,McConnell2004,Stevens2000,Brown1998}, hardware \cite{WikipediaContributors2018m,Faes2010,Haynal2009,Faes2008,Haynal2008,Haynal2008a,Zeng2006,Zeng2005,Keating2002,Keating1999}, and/or system \cite{Collier2012,Rothman2007,Laplante2004} architecture as well as databases \cite{Ambler2006}. In terms of personal and professional development, collaborators are strongly encouraged to refactor their wetware \cite{Hunt2008} and reduce their personal technical debt \cite{Levin2017,Aboulafia2016,Davis2013a,Prause2011,Gadoury2010}, too. \\

We shall also use project portfolio management \cite{Rothman2016a,Rothman2009} to help us manage projects that we are involved in. \\

Moreover, for projects involving integrated circuit design (and embedded hardware or cyber-physical system design), morph your process for VLSI CAD engineering into hardware DevOps (hardware/IC development and information-technology operations) \cite{MythicIncStaff2018}.












%	Read the following PDF files:
%Rothman2009
%Rothman2007
%Hunt2008
%Derby2006
%Davies2009
%Hunt2006
%Jeffries2015
%Plattner2011a
%AnacondaStaff2019
%Project Portfolio Tips: Twelve Ideas for Focusing on the Work You Need to Start & Finish
%Subramaniam2006.pdf








%%%%%%%%%%%%%%%%%%%%%%%%%%%%%%%%%%%%%%%%%%%
\subsection{Practice of Automated Regression Testing}
\label{ssec:PracticeOfAutomatedRegressionTesting}

Regarding the practice of automated regression testing, Mr. Heiko Maurer (then a lecturer at the University of Adelaide) and Dr./Mr. Tishampati Dhar (a former classmate at the University of Adelaide) suggests printing information regarding passed test cases to a file (or to standard output) and printing information regarding failed test cases to another file (or to standard error). During build automation of software, such as {\tt gem5} \cite{gem5developers2014,Binkert2011}, carry out automated (regression) testing during the last stage of the build/installation process to ensure that the build/installation was done correctly. When performing automated software testing (or software test automation), list the the test cases and their test results (i.e., ``OK''/``Fail''), just like {\tt gem5} during the testing phase of build automation. At the end of each automated software testing run (or round/run of automated software testing), indicate the total number of test cases used, the total number of test cases passed, and the percentage of test cases passed (with respect to the total number of test cases used). \\


\cite[\S Testing Guidelines]{TheSciPyCommunity2019c} provides a set of testing guidelines for *Python* libraries and packages, or software in general.





%%%%%%%%%%%%%%%%%%%%%%%%%%%%%%%%%%%%%%%%%%%
\subsection{Cloud-based Data Science and Machine Learning}
\label{ssec:CloudBasedDataScienceAndMachineLearning}

Regarding data science projects, we shall use a lot of the aforementioned software development practices and methodologies \cite{Cady2017}. Also, we can use cloud-based machine learning (and deep learning) platforms, such as Google's {\it Colaboratory} \cite{GoogleColabStaff2019}, Anaconda's {\it Anaconda Cloud} \cite{AnacondaStaff2019}, and Amazon's {\it Amazon Web Services} (AWS) \cite{AmazonWebServicesStaff2019}. These cloud-based software services, also known as software as a service (SaaS), helps to bring data science, machine learning, and deep learning capabilities to more people, since they do not need expensive, modern hardware to run computationally intensive tasks for data analytics and machine learning.








%%%%%%%%%%%%%%%%%%%%%%%%%%%%%%%%%%%%%%%%%%%
\section{Additional Guidelines}
\label{sec:AdditionalGuidelines}

Please kindly use the {\it Markdown} language for writing text documents. This is because {\it Bitbucket} will treat my text file as a file written in the {\it Markdown} syntax. That said, the raw file looks a lot better than the represented {\it Markdown} files. Their ({\it Bitbucket}) formatting for {\it Markdown} is messed up. {\it GitHub}'s formatting for {\it Markdown} works as expected. To insert mathematical expressions into {\it Markdown} documents, use {\it TexPaste} \cite{Nguyen2013} to typeset the mathematical expressions via \LaTeX\ and insert snapshots of these mathematical expressions as pictures in the {\it Markdown} documents. A recommended style guide for {\it Markdown} is from {\it Google} \cite{Turakulov2018}. \\

In addition, tools for working with source code and \LaTeX\ source files include: \vspace{-0.3cm}
\begin{enumerate} \itemsep -4pt
\item {\tt git} (or Git) \cite{Hamano2014,Driscoll2012}: \vspace{-0.3cm}
	\begin{enumerate} \itemsep -2pt
	\item {\tt Mercurial SCM} \cite{MercurialContributors20XY,MercurialContributors20XYa,MercurialContributors2019,MercurialContributors2013,MercurialContributors2013a,DeMare2015,OSullivan2009,AtlassianStaff2018a} can be used as a substitute.
	\end{enumerate}
\item {\tt latexdiff}: ``determine and markup differences between two latex files'' \vspace{-0.3cm}
	\begin{enumerate} \itemsep -2pt
	\item Evan Driscoll, ``Latexdiff notes,'' from {\it Evan Driscoll's Web page: Writings on Software: \LaTeX}, the Department of Computer Sciences, University of Wisconsin-Madison College of Engineering, University of Wisconsin-Madison, Madison, WI. Available online at: \url{http://pages.cs.wisc.edu/~driscoll/software/latex/latexdiff.html}; last accessed on February 15, 2016 \cite{Driscoll20XYb}.
	\end{enumerate}
\item documentation generators: \vspace{-0.3cm}
	\begin{enumerate} \itemsep -2pt
	\item {\tt Doxygen} \cite{vanHeesch2016}
	\item {\tt Texinfo}-based generators \cite{WikipediaContributors2016h,Stallman2016,Stallman2016a,Stallman2016b}
	\end{enumerate}
\item Build automation: \vspace{-0.3cm}
	\begin{enumerate} \itemsep -2pt
	\item {\tt SCons} \cite{Driscoll20XY}
	\end{enumerate}
\end{enumerate}


Data sets and sets of benchmarks for experiments shall be publicly published using an online repository, via {\it figshare LLP} \cite{figshareLLPstaff2016} and/or {\it DataHub} \cite{DatopianAtomaticLtdIncStaff2017}. For each data set, or each set of benchmarks, create a unique Digital Object Identifier (DOI) \cite{InternationalDOIFoundationStaff2017} to identify it. \\

Repositories for software as well as designs of integrated circuits and cyber-physical systems shall be stored online, using online repositories such as {\it GitHub} \cite{GitHubStaff2016}. Each repository shall have a unique DOI to identify it, and include all source code, documentation, and design files. There also exists cloud-based repositories for the source code of software/hardware projects that allow me to execute my software (or simulate my hardware). E.g., see \cite{CodeOceanstaff2017,RunMyCodeAssociationContributors2013} as examples to facilitate research reproducibility, replicability, and repeatability. This supports research reproducibility and reproducible research \cite{Schiermeier2018,Baumer2017,CodeOceanstaff2017,DatopianAtomaticLtdIncStaff2017,Kim2017,Mailund2017,Barba2016,Gandrud2015,Liberman2015,Creswell2014,Gandrud2014,Stodden2014,RunMyCodeAssociationContributors2013,Blackburn20XY,Geier20XY,Krishnamurthi20XY}. \\

Please kindly note that {\it GitHub} \cite{GitHubStaff2016}: \vspace{-0.3cm}
\begin{enumerate} \itemsep -4pt
\item Does not allow a {\it GitHub}-based page to be refreshed/reloaded many times in a few seconds. Else, it would report the following: \vspace{-0.3cm}
	\begin{enumerate} \itemsep -2pt
	\item ``Whoa there!''
	\item ``You have triggered an abuse detection mechanism.''
	\item ``Please wait a few minutes before you try again.''
	\end{enumerate}
\end{enumerate}



If possible, develop {\it Python} software in a {\it Python}ic style \cite[Chapter 1, pp. 1--12,12--17]{Alchin2010} \cite{Jon2014,Reitz2016a,Reitz2016,Preshing2014,Franca2014,vanRossum2013}. \\



As aforementioned in \S\ref{ssec:AdditionalRecommendationsForManagingASharedBibTeXDatabase}, when copying (and pasting) text from a document or a Web page, non-ASCII hidden characters may be accidentally copied from the source (document or Web page) to the source code for a computer language. Similarly, corruption in memory subsystems or storage devices (such as a solid-state drives) and data transfer between computers can result in corrupting text files to produce non-ASCII hidden characters in these files. Therefore, if these text files use a particular computer language, use parsers (for software such as compilers, interpreters, electronic design automation software, or other software applications) for these languages to detect/find these non-ASCII hidden characters.



%%%%%%%%%%%%%%%%%%%%%%%%%%%%%%%%%%%%%%%%%%%
\subsection{Suggested Prefixes for Labels of Parts of a \LaTeX\ Document}
\label{ssec:SuggestedPrefixesForLabelsOfPartsOfALaTeXDocument}



Add one of the following prefixes to labels in my \LaTeX\ documents for parts, such as chapters, sections, subsections, subsubsections, figures, tables, equations, code listings, definitions, theorems, lemmas, corollaries, propositions, proofs, examples, and remarks: \vspace{-0.3cm}
\begin{enumerate} \itemsep -4pt
\item ``chp:'' for chapter
\item ``sec:'' for section
\item ``ssec:'' for subsection
\item ``sssec:'' for subsubsection
\item ``fig:'' for figure
\item ``tab:'' for table
\item ``eqn:'' for equation
\item ``lst:'' for code listing
\item ``defn:'' for definition
\item ``thrm:'' for theorem
\item ``lem:'' for lemma
\item ``crly:'' for corollary
\item ``prop:'' for proposition
\item ``prf:'' for proof
\item ``eg:'' for example
\item ``rem:'' for remark
\end{enumerate}


This practice makes it easier for collaborators and paper/reviewers to determine if the label for a part of the \LaTeX\ document, belongs to a label or refers to something else. This should be done with cross-referencing different parts of my \LaTeX documents to make it easier for readers to follow the storyline, and look up information.














%%%%%%%%%%%%%%%%%%%%%%%%%%%%%%%%%%%%%%%%%%%
\section*{Acknowledgments}
\label{sec:Acknowledgments}
\addcontentsline{toc}{section}{Acknowledgments}


Mr. David Knight (then a lecturer at the University of Adelaide) and Dr. Charles Lakos (then a senior lecturer at the University of Adelaide) introduced me to regression testing and automated software testing during their introductory course on software engineering. During programming assignments and projects for this course, Dr. Nikolay Stoimenov helped me honed my skills in regression testing and automated software testing, via the practice of pair programming \cite{DeOrio2016,Fox2013,Oram2011,Jalote2008,Shore2008,Wiegers2002}. Subsequently, Mr. Heiko Maurer (then a lecturer at the University of Adelaide) planted the seeds of automated regression testing with his suggestion of separating the results of test cases that passed from the results of test cases that failed. Shortly after, Dr./Mr. Tishampati Dhar (a former classmate at the University of Adelaide) suggests printing information regarding passed test cases to a file (or to standard output) and printing information regarding failed test cases to another file (or to standard error). Months later, Dr. Francis Vaughan (then a senior lecturer at the University of Adelaide), Mr. Kevin J. Maciunas (then a lecturer at the University of Adelaide), and Dr. Robert Esser (then a senior lecturer at the University of Adelaide) helped me develop a sound methodology towards automated regression testing. In addition, Dr. Lakos and Dr. Esser introduced me to using formal methods and software formal verification in the software development process.

% [Redacted for clarity; removed references for pair programming]
%	Mr. David Knight (then a lecturer at the University of Adelaide) and Dr. Charles Lakos (then a senior lecturer at the University of Adelaide) introduced me to regression testing and automated software testing during their introductory course on software engineering. During programming assignments and projects for this course, Dr. Nikolay Stoimenov helped me honed my skills in regression testing and automated software testing, via the practice of pair programming. Subsequently, Mr. Heiko Maurer (then a lecturer at the University of Adelaide) planted the seeds of automated regression testing with his suggestion of separating the results of test cases that passed from the results of test cases that failed. Shortly after, Dr./Mr. Tishampati Dhar (a former classmate at the University of Adelaide) suggests printing information regarding passed test cases to a file (or to standard output) and printing information regarding failed test cases to another file (or to standard error). Months later, Dr. Francis Vaughan (then a senior lecturer at the University of Adelaide), Mr. Kevin J. Maciunas (then a lecturer at the University of Adelaide), and Dr. Robert Esser (then a senior lecturer at the University of Adelaide) helped me develop a sound methodology towards automated regression testing. In addition, Dr. Lakos and Dr. Esser introduced me to using formal methods and software formal verification in the software development process.


%%%%%%%%%%%%%%%%%%%%%%%%%%%%%%%%%%%%%%%%%%%%%
%%%%%%%%%%%%%%%%%%%%%%%%%%%%%%%%%%%%%%%%%%%%%
%
%	End of document
%
%	Inserting references
%
%%%%%%%%%%%%%%%%%%%%%%%%%%%%%%%%%%%%%%%%%%%%%
%%%%%%%%%%%%%%%%%%%%%%%%%%%%%%%%%%%%%%%%%%%%%
%	Beginning of BACK MATTER: bibliography, indexes and colophon
%\backmatter

%%%%%%%%%%%%%%%%%%%%%%%%%%%%%%%%%%%%%%%%%%%%%
{\linespread{1}
\bibliographystyle{plain}
%\bibliography{/Users/zhiyang/Documents/ricerca/antipastobibtex/references}
%\bibliography{/Users/zhiyang/Documents/ricerca/lassi-bibtex/references}
\bibliography{/Users/zhiyang/Documents/ricerca/saag-bibtex/references}
}
%\bibliography{/data/research/antipastobibtex/references}
%%%%%%%%%%%%%%%%%%%%%%%%%%%%%%%%%%%%%%%%%%%%%
\end{document}